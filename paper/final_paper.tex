\documentclass{article}
\usepackage{titlepic}
\usepackage{graphicx}
\usepackage{float}
\usepackage{hyperref}
\title{Method to the Madness: An Analytical Approach to Bracketology}

\author{Michael Egle\\
2022 Carnegie Mellon Sports Analytics Conference}

\begin{document}
\maketitle

\begin{abstract}

Every year in March, college basketball teams from across the country tune into the NCAA Tournament Selection Show. For better or for worse, some teams already know their fate. Other teams are left to stress about whether they’ll hear their name called. The phrase “bracketology” was coined to describe the process of selecting and seeding the field of 68 teams. Some media outlets post daily bracketology projections during the regular season, giving fans an idea of where their team stands. In this paper, I set out to calculate day-by-day probabilities of making the tournament and expected seed for each team in Division 1 men’s and women’s basketball. These projections are based on the variables that the NCAA Tournament Selection Committee references when creating the bracket. These variables include the team’s record, overall strength of schedule, quality wins, and more. The projections are visible via a web app that I have created to make this project more interactive. I also look to close the existing gap between men’s and women’s basketball in bracketology coverage. While there is no shortage of bracketology projections for the men’s tournament, women’s projections are much harder to find, especially through an analytical lens.

\end{abstract}

\section{Introduction}

\subsection{Bracketology}



\end{document}